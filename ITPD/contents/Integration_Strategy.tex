\section{Entry criteria}
Specify the criteria that must be met before integration testing of specific elements may begin (e.g., functions must have been unit tested).

\section{Elements to be integrated}
Identify the components to be integrated, refer to your design document to identify such components in a way that is consistent with your design.

\section{Integration testing strategy}
Describe the integration testing approach (top down, bottom up, functional groupings, etc.) and the rationale for the choosing that approach.

\section{Sequence of Component/Function Integration}
NOTE: The structure of this section may vary depending on the integration strategy you select in Section 2.3. Use the structure proposed below as a non mandatory guide.

	\subsection{Software Integration Sequence}
	Identify the sequence in which the software components will be integrated within the subsystem. Relate this sequence to any product features/functions that are being built up.
	\subsection{Subsystem Integration Sequence}
	Identify the order in which subsystems will be integrated.

If you have a single subsystem, 2.4.1 and 2.4.2 are to be merged in a single section. You can refer to Section 2.2 of the test plan example [1] as an example of what we expect.
