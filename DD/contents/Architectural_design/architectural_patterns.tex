Building this application different architectural styles and patterns have been used.
\subsection{4-tier JEE client-server architecture}
	This architectural style has been used for separating efficiently the different levels of execution. The components of the application are identified in this tiers:
	\begin{itemize}
		\item{Client tier: is the layer that interact with the users. It runs on the client machine. This layer contains the On-Board computer, the Mobile application and the Web application. 
		\\The Mobile application can have three different architectures: iOS, Android and WP. It interacts with the Application server using the RESTful API.
		\\The On-Board computer has got a proper application. It need to be developed for the specific type of car in order to interact with the car's equipment and with the specific sensors used by PowerEnJoy. It also interacts with the Application server using the RESTful API.
		\\Finally the Web application interacts with the Web server using the RESTful API too.}
		\item{Web tier: is implemented using the JEE 7. The Web server implementation is GlassFish Server, as the Business tier, in order to avoid conflicts. It contains Servlets, JavaServer Pages (JSP) and RESTful API. The MVC pattern is implemented in the web logic thanks to the JPS. The Servlets are useful only in specific interactions while the RESTful Api are used in the interaction with the Application server.}
		\item{Business tier: is implemented using the JEE 7, commonly used in large-scale projects. The application server implementation is  GlassFish Server. It contains Enterprise Java Beans(EJB), Java Persistence API (JPA), JAX-RS and RESTful API. Thanks to JAX_RS implements RESTful APIs to interface with clients and the Web tier, and receives data from them. The data are processed (if necessary) using the EBJs and then the data are sent to the Database for storage by the JPA that access to the DB and execute the object-relation mapping. The RESTful APIs are used to interact with external systems.}
		\item{Enterprise Information System tier: runs EIS software and includes the enterprise infrastructure systems It represents the data layer, where the data are stored and retrieved by the Application server. MySQL is the relational DBMS chosen for the creation and the maintenance of all the application data.}
	\end{itemize}
\subsection{Client-Server}
	The client-server communication model is highly used in this application.
	The On-Board computer and the Mobile application are clients with respect to the Application Server. The Web application consist of a Web browser that is a client with respect to the Web Server. The Web server is also a client with respect to the Application server. The Database is a server with respect to the Application server that act as a client.
\subsection{Thin client}
	In order to avoid that the client machine is involved in any logic decision we decided that all the computations will be run in the Application server. This comports different important advantages for the client tier: the client application will comport lower operational cost for the device, a superior security is obtained, the data are synchronized and the system is highly reliable. In addition this make the application independent from the number of clients connected.
\subsection{MVC}
	The Model-View-Controller pattern has been used in this application during the implementation of the client tier. In this way we separated the model, that rapresent the knowledge, the view, that is a visual representation of the model, and the controller, that is the link between a user and the system.
