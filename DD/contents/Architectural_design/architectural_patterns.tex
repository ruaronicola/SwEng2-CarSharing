Building this application different architectural styles and patterns have been used.
\subsection{4-tier JEE client-server architecture}
	This architectural style has been used to separate efficiently the different levels of execution. The 4 tiers componing the application are:
	\begin{itemize}
		\item{\textbf{Client tier}: is the layer that interact with the users, it runs on the client machine(On-Board computer, Mobile application, Web application). 
		\\The Mobile application interacts with the Application server using the RESTful API and will be deployed for three different architectures: iOS, Android and WP.
		\\The On-Board application interacts with the vehicle's sensors through the vehicle-specific API. It also interacts with the Application server using the RESTful API.
		\\Finally the Web application interacts with the Web server using the RESTful API too.}
		\item{\textbf{Web tier}: is implemented using JEE 7 and GlassFish Server in order to avoid conflicts. It contains Servlets, JavaServer Pages (JSP) and exposes a RESTful API. The MVC pattern is implemented in the web logic thanks to the JSP. The Servlets are useful only in specific interactions while the RESTful Api is used in the interaction with the Application server.}
		\item{\textbf{Business tier}: is implemented using JEE 7 and GlassFish Server, commonly used in large-scale projects. It contains Enterprise Java Beans(EJB), Java Persistence API (JPA) and JAX-RS which exposes a RESTful API to interact with clients and the Web tier. The data are processed (if necessary) using the EJBs and then sent to the Database by the JPA that access the DB and executes the object-relation mapping. RESTful APIs are used to interact with external systems.}
		\item{\textbf{Enterprise Information System tier}: runs EIS software and includes the enterprise infrastructure systems. It represents the data layer, where the data are stored and retrieved by the Application server. MySQL is the relational DBMS (rDBMS) chosen for the creation and the maintenance of all the application data.}
	\end{itemize}
\subsection{Client-Server}
	The client-server communication model is highly used in this application.
	\\The On-Board computer and the Mobile application are clients with respect to the Application Server. The Web application consist of a Web browser that is a client with respect to the Web Server. The Web server is also a client with respect to the Application server. The Database is a server with respect to the Application server that act as a client.
\subsection{Thin client}
	In order to avoid the involvement of the client machine in any logical decision all the computations is done in the Application server. This comports different important advantages for the client tier: the client application has lower operational cost for the device, a superior security is obtained, the data are synchronized and the system is highly reliable. In addition this makes the application independent from the number of clients connected.
\subsection{MVC}
	The Model-View-Controller pattern has been used in this application during the implementation of the client tier. In this way we separated the model (which represent the knowledge) the view (which is a visual representation of the model) and the controller (which is the link between a user and the system).
