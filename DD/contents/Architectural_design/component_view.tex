\begin{figure}[!ht]
    \centering
    \vspace{0.2cm}
    \includegraphics[width=1.0\textwidth]{/DD/component_view}\\
    \vspace{0.4cm}
    \caption{Component view for the Application Server} 
    \label{fig:component_view} 
\end{figure}

\subsection{System components}
To define and easily understand what kind of functionalities must be implemented in our system we decided to decompose PowerEnJoy logically into components, which are reusable and easily adaptable bricks for our application. 
\\In this section the single components and their interactions are analysed. 

\begin{itemize}
	\item \textbf{AuthenticationManager}: This component provides all the authentication-related functionalities such as registration, login and credentials generation. It is important to remark that a RESTful API is provided and so no session is created, instead a token is provided and used for authentication purposes.
	\item \textbf{ProfileManager}: This component manages all the profile-related functionalities in order to allow informations' editing.
	\item \textbf{LocationManager}: This component handles the logic behind the vehicle/user localization and tracking, it is also responsible for safe-areas and charging-stations' location consistency.
	\item \textbf{ReservationController}: This component manages the reservation logic, it receives informations from the LocationManager, correctly handles the timing for expiration and queries the CarManager component to update the car status (FREE, RESERVED, INUSE, OUTOFSERVICE). It is responsible for the Reservation logic and correctness checking.
	\item \textbf{RideController}: This component controls the (un)locking of the car, the car status and correctly handles the timing and charges for the ride. It is responsible for the Ride logic and correctness checking.
	\item \textbf{CarManager}: This component is responsible for communications with the on-board computer and for car's status update.
	\item \textbf{ChargeManager}: This component handles the application of charges for rides and reservations, it also process the applications of fees and discounts due to bad/virtuous behaviours. It is responsible to communicate with the PaymentGateway to complete the payment process.
	\item \textbf{NotificationManager}: This component manages the users' notification, in particular regarding charges and payment requests. It communicates with the NotificationGateway to effectively notify the users.
	\item \textbf{PaymentGateway}: This component is responsible for the communication with the external payment handler in order to effectively request the payments(automatic payments are pre-authorized).
	\item \textbf{NotificationGateway}: This component actually creates and send the user notification.
	\item \textbf{Router}: This component is responsible for routing the requests to the correct components.
	\item \textbf{Client}: The actual client device(Mobile/Web application).
	\item \textbf{Model}: The data we interact with, this is an abstraction of the DataBase.
	\item \textbf{DataBase}: The database used to store persistent data.
\end{itemize}

\subsection{Database components}
In particular the data stored in the database will be split through different subcomponents that identified the main entities of our system:
User,Vehicle, Location, Safe Area, Charging Station, Reservation, Ride, Behaviours and Payment.  
\\The designed model for persistent data is provided here in a ER diagram in order to better analyze the motivations of our design. That's the representation of the database model:
\begin{figure}[!ht]
  \centering
  \vspace{0.2cm}
  \includegraphics[width=1.0\textwidth]{/DD/ER_Diagram}\\
  \vspace{0.4cm}
  \caption{ER Diagram} 
  \label{fig:ER_Diagram} 
\end{figure}

 And this is the relation schema associated with the ER diagram.
\begin{itemize} 
	\item{User (\underline{Username}, \textit{Location}, e-mail, password, driving\_license\_number, name, surname, address, phone number, taxcode, card\_number, expiration\_date, card\_security\_code, card\_holder\_name, status)} %status: reserving, riding, free, (guarda caso utente bloccato)
	\item{Vehicle (\underline{ID\_Vehicle}, \textit{Location}, plaque, type, odometer, battery\_level, status, num\_of\_seats, num\_of\_passengers) }
	\item{Location (\underline{coordinate}, city)}	
	\item{Safa Area (\underline{ID\_SA}, \textit{Location})} %type A=Safe Area, B=Charging Station, C=both, Add an ID!!!
	\item{Charging Station(\underline{ID\_CS}, \textit{Location},\textit{ID\_SA})}
	\item{Reservation (\underline{ID\_reservation}, \textit{ID\_User}, \textit{ID\_Vehicle}, start\_date\&time, end\_date\&time, expired)}
	\item{Ride (\underline{ID\_Ride}, \textit{ID\_User}, \textit{ID\_Vehicle}, \textit{ID\_Reservation}, \textit{ID\_Payment},\textit{Start\_Location},\textit{Finish\_Location}, start\_date\&time, end\_date\&time, status, num\_of\_passengers, travel\_distance, battery\_usage)}
	\item{Behaviours(\underline{Type}, \textit{ID\_Ride}, price\_variation)}
	\item{Payment (\underline{ID\_Payment}, \textit{ID\_User}, \textit{ID\_Reservation}, \textit{ID\_Ride}, date\&time, total\_cost, status, type)}
\end{itemize}
In the User entity there are all the main information about the user such as credentials and payment method. Her status can be reserving, riding, free or banned. The location of the user is not necessary.
\\A Vehicle entity has different attributes, some of them supply by the On-Board computer. The status here can be FREE, RESERVED, INUSE or OUTOFSERVICE.
\\The Location entity represent a location provided by the GPS. Safe area are zones composed by one or more locations. Into the Safe areas there can be also some Charging stations.
\\The Reservation entity is connected with an user and a vehicle. It can end or with an unlock or with an expiration, and in this second case it comports the payment of a fee by the user. The reservation is strictly connected with the Ride entity, which can exist only related with it. Each ride has a starting point and when it finishes an end point. A ride also comports a payment, in particular if the Behaviours entity is present related with the ride it comport a variation of the final price. 
\\In the Payment entity there are the information about the transaction from th user to PowerEnJoy, the status indicates if the payments has been done with success or if the system is still waiting for it.
