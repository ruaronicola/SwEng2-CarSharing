\subsubsection{Functional Requirements}
\begin{itemize}
  \item The system shall kwon if a car is avaible or not.
  \item The system shall know if an user is already reserving a car.
  \item The system shall turn a car avaible or not avaible.
  \item The system shall know the time when an user do a reservation.
  \item The system shall tag a car avaible again if a car is not picked-up within an hour from the reservation.
\end{itemize}

\subsubsection{Scenario 1}
Mark needs to go to the city center in the late evening and decides to use a PowerEnJoy's car. Just before having a quick lunch in a near restaurant he browses the application and reserves car near him to make sure that it will still be available when he'll be back.
Unexpectedly Morgan, a friend of Mark, is lunching in the same restaurant and when they see each other they start chatting about everything and anything. When they stop talking Mark notices an e-mail from PowerEnJoy: it's about his reservation that has expired. His bank has also notified him that a fee of 1€ was applied by PowerEnJoy.


\subsubsection{Scenario 2}
Susan is a typical PowerEnJoy user, she was walking through a street when a friend texts her to hang out. Fortunately she saw a PowerEnJoy's car on the other side of the street. She reached the car and checked online: that's not available. Someone had probably already reserved it. But on the app she noticed that there were two avable car 5 minutes walk from her position. Than she reserved one of that two cars and easily reached it, thanks to the accurate location service provided by the system. 


\subsubsection{Mockups}
\begin{figure}[!ht]
  \centering
  \vspace{0.1cm}
  \includegraphics[width=1\textwidth]{/RASD/System_Functions/create_reservation_mockup}\\
  \vspace{0.4cm}
  %\caption{Mockup for the login mobile page} 
  \label{fig:create_reservation_mockup} 
\end{figure}
% maybe home page mockup here 


\subsubsection{Use-case table}
\begin{center}
  \begin{tabular}{ l | p{10cm} }
    \hline
    Actors & Guest\\ \hline
    Goal & G\ref{itm:goal-reservation}\\ \hline
    Entry conditions & \begin{itemize}
			\item The User is in his homepage and want to reserve a car.
			\item PowerEnJoy has available cars.
			\item The User is not already reserving a car.
\end{itemize}  \\ \hline
    Flow of events &
\begin{itemize}
%\item The User select the "Find a car" option.
\item The User has three ways for finding a car:
\begin{itemize}
			\item browse the map.
			\item use her location.
			\item enters an address.
\end{itemize}
%\item The system loads a map of the selected area and the avaible cars' locations near there .
\item The User selects the car that want to reserve.
\item the system displays the car's information.%Maybe to add to the mockup
%\item The User presses the "Yes,I'm sure" button for confirm the reservation.
\item The system turn the car "reserved".% and start the reservation timer.
\item The system notified the User.
\end{itemize} \\ \hline
    Exit conditions &
\begin{itemize}
	\item A car is set as "reserved".
	\item The car is reserved for up one hour. After that the reservation expires and the User has to pay a fee of 1€.
\end{itemize}  \\ \hline
  Exceptions & 
%\begin{itemize}
%\item The User has already reserved a car.

%\item 
The system is not able to complete the operation due to some internal issues or connection broken (the system signals a ConnectionFailure).%volendo si possono modificare i nomi delle eccezzioni.
%\end{itemize} 
\\ \hline
  \end{tabular}
\end{center}


\subsubsection{Sequence diagram}
\begin{figure}[!ht]
  \centering
  \vspace{0.2cm}
  \includegraphics[width=1.2\textwidth]{/RASD/System_Functions/reservation_sequence}\\
  \vspace{0.1cm}
  %\caption{Sequence diagram for the login procedure} 
  \label{fig:reservation_sequence} 
\end{figure}

