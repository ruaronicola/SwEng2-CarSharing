\subsubsection{Functional Requirements}
\begin{itemize}
  \item The user has to provide her \gls{ID}.
  \item The user has to provide her \gls{pwd}.
\end{itemize}

\subsubsection{Scenario 1}
Susan has just registered to PowerEnJoy and loads the home page in order to login, she than enters her credentials and submit the form. The credential are checked and Susan is redirected to the web application as a logged user.

\subsubsection{Scenario 2}
Mark is a typical PowerEnJoy user, he got used to the credentials form submission and types his password quickly. The password is not recognized by the system and Mark is asked to check his credentials. He than types more accurately and finally gets redirected to the web application. % todo: maybe forgot password

\subsubsection{Scenario 3}
A guest confused the login form with the registration one and after submiting the form is told that such user is not registered to the PowerEnJoy service. She than clicks on "Register to PowerEnJoy" and follows the registration procedure.

\subsubsection{Use-case table}
\begin{center}
  \begin{tabular}{ l | p{10cm} }
    \hline
    Actors & Guest\\ \hline
    Goal & G\ref{itm:goal-access}\\ \hline
    Entry conditions & 
\begin{itemize}
\item The Guest has to be sucessfully registered to PowerEnJoy.
\item The Guest entres the Log-in page in the browser/ mobile application. 
\end{itemize} \\ \hline
    Flow of events &
\begin{itemize}
\item The Guest insert her \gls{ID} and \gls{pwd}.
\item The Guest clicks the Log-in button.
\item The system loads the User's homepage.
\end{itemize} \\ \hline
    Exit conditions & The User is in the PowerEnJoy homepage. \\ \hline
  Exceptions & 
\begin{itemize}
\item The Guest provides wrong username-password pair (the system signals a LoginError).
\item The Guest doesn't fill both the fields (the systems signals an InformationLack).
\item The system is not able to complete the operation due to some internal issues or connection broken (the system signals a CennectionToSystemFail).%volendo si possono modificare i nomi delle eccezzioni.
\end{itemize} \\ \hline
  \end{tabular}
\end{center}
