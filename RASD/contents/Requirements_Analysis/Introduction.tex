\section{Purpose of the system}
The purpose of this project is to develop a digital management system called PowerEnJoy. 
\\PowerEnJoy is a car-sharing service which uses only eletric-cars and allow the users to easily find a car, thanks to the location services, and to use it. 
 % TODO: Maybe "A previous version of the system doesn't exist so it will be created completely from scratch." OR "The system is completely new and nothing has been implemented before."

\section{Scope of the system}
Users have to be registered to the system and provide credentials (as well as payment informations), then a password will be sent that can be used to access the car-sharing service. They can easily find eletric-cars thanks to the location services and reserve them for up to an hour. 
\\When an user is near to a vehicle that she want to drive, she contacts the system telling that she's nearby that specific car, than the system unlock the car letting her to get into it. 
Then the system automatically calculates the charge during the ride, notifying the user through a screen on the car. 
Finally when the car is parked in a safe area and the user exits the car, the system stops charging the user and automatically lock the car that become available again.
\\The system should encourage virtuous behaviour of the users and to do that some discounts can be applied on their last ride. For example a discount of the 10\% is applied if the user took at least two other passengers onto the car. 
\begin{comment}
Other discounts are applied if a car is left with no more than 50\% of the battery empty or if the user left the car in a special park where it can be recharged(20\%) 
and she takes care of plugging the car into the power grid (30\%). 
\\On the other side the system charges 30\% more on the last ride if the car is left at more than 3 Km from the nearest power grid station or with more than 80\% of the battery empty in order to compensate for the cost required to re-charge the car on-site.
% TODO: Point e is not necessary. MAYBE DOT-LIST HERE
\end{comment}

\section{Objectives and success criteria of the project}
\label{sec: proj_objectives}
After a first analysis on our average users, the main features that should be provided by our application are:
\begin{enumerate}
	\item{[G\ref{itm:goal-registration}] User's registration}\label{itm:goal-registration}
	\item{[G\ref{itm:goal-access}] User's account and session management}\label{itm:goal-access}
	\item{[G\ref{itm:goal-payments}] Immediate payments/charges}\label{itm:goal-payments}
	\item{[G\ref{itm:goal-location}] Location-related services (car localization: user's location or address should be provided)}\label{itm:goal-location}
	\item{[G\ref{itm:goal-reservation}] Car real-time reservation and related processing}\label{itm:goal-reservation}
	\item{[G\ref{itm:goal-locking}] Car un-locking and (automatic) locking}\label{itm:goal-locking}
	% TODO: Maybe other more system-related features
\end{enumerate}

\section{Actors Identification}
\begin{itemize}
	\item{{\bf Guest}: is a person that hasn't already registered to the system or an user that hasn't already logged in. She can only proceed with a new registration or log in.}
	\item{{\bf User}: we believe that PowerEnJoy can be used by a wide range of people that need only to access the system for benefit. A person became an user after her registration to the system, when she provides her credentials and payment information. 
	\\A tipical user is a person who want to easily move around in a social end eco-friendly way, usually only for short travels near the \glspl{charging station}.
	\\Our scope is to create an easy-to-use and efficent system that make the users satisfied and willing to use PowerEnJoy.}
\end{itemize}

\section{Stakeholders Identification}
\begin{itemize}
	\item{{\bf Customers}: the purpose of the customer is to maximise the performances of the system and reach the biggest profit possible.
	\\In our project the main stakeholder is the professor. She expects us to develop a digital menagment system, PowerEnJoy, that provide the functionality normally provided by car-sharing services.  
	\\The role of the professor is to evaluate our ability and level of comprehension of the subject.}
	\item{{\bf Producers}: the project is developed and produced by us. We want to apply in practice what we learn during lectures and became able to manage software engineering problems, solving them in a rigorous way and being as accurate as possible. Our project includes a Requirement Analysis and Specification Document (RASD), a Design Document (DD), a testing-related activity, an
	assessment of the effort and cost required for the development of the project, a code inspection and bug identification activity on an existing open souce project. 
	\\We will develop our project as close as possible to a real application that is ready to be launched in the market.}
\end{itemize}
