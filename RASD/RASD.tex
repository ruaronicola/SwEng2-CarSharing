\documentclass[openright]{report}

% Auxiliary variable for version tracking
\def \version {0.1}

% Including titlesec package 
% documentation: www.ctex.org/documents/packages/layout/titlesec.pdf
\usepackage{titlesec}
\titleformat{\chapter}{\Huge\bfseries}{}{0pt}{\Huge}

% Including graphicx package for images
\usepackage{graphicx}
\graphicspath{{../resources/images/}}

%Include tocbibind package for bibliography in TOC
\usepackage{tocbibind}


\begin{document}
	\begin{titlepage}
		\centering
		\includegraphics[width=0.50\textwidth]{polimi}\\\vspace{0.25cm}
		{\scshape\LARGE Politecnico di Milano\par}\vspace{0.25cm}
		{\scshape\Large Software Engineering II project: PowerEnjoy\par}\vspace{1.5cm}
		{\huge\bfseries Requirements Analysis and Specification Document\par}\vspace{1cm}
		{\large Gregori Giacomo and Ruaro Nicola\par}\vfill

		% Bottom of the page
		{\large \today \\Version \version}
	\end{titlepage}

    
    \tableofcontents

    \begin{abstract}
		The main purpose of this document is to give a specification of the requirements that our system has to fulfill adopting the IEEE-830 standard for RASD documentation. 
		It also introduces functional and non-functional requirements via high level specification of the system. 
		In the last part a formal model is presented using Alloy. 
		\\The information contained in this document is intended for the stakeholders and developers: for the stakeholders this document presents an useful description to understand the project development, meanwhile for the developers it’s quite a comfortable way to match the stakeholders' requests and the proposed solutions.
	\end{abstract}

    \part{Requirements analysis}
    \chapter{Introduction}
    \section{Purpose}
	The aim of this project is to develop a digital management system called PowerEnJoy. That is a car-sharing service which uses only eletric-cars. It permits to the user to easily find a car thanks to the location provided and to use it. For incentivize the virtuous behaviors of the users there are some discounts and some lots that can be applied to the bill.
		%A previous version of the system doesn't exist so it will be created compleately here.
		%(more or less)
	\section{Scope}
	Users have to be registered to the system providing credentials as well as payment informations and they have back a password that can be used for acess to the car-sharing service. Than they can easily find the positions of the eletric-cars thanks to the location service provided and reserve them for up to an hour. 
	\\When an user is near to a vehicle that want to drive he contacts the system telling it that he's nearby that specific car, than the system unlock the car letting him to get into it. Then the system automatically provide to calculate the charge during the travel, notifying it to the user through a screen on the car. Finally when the car is parked in a safe area and the user exits the car the system stops to charge the user and automatically lock the car that become available again.
	\\The system should incentivize a virtuous behaviour of the users, %(to change if possible)
		and to do that it can applies some discounts on their last ride. For example a discount of the 10\% is applied if the user took at least two other passengers onto the car. Other discounts are applied if a car is left with no more than 50\% of the battery empty (20\%) or if the user left the car in a special park where it can be recharged, and he take care of plugging the car into the power grid (30\%). On the other side the system charges 30\% more on the last ride if the car is left at more than 3 Km from the nearest power grid station or with more than 80\% of the battery empty due to compensate for the cost required to re-charge the car on-site.
		%point e not necessary.
    
    \section{Goals}
		[Sample text]
	\section{Actors}
		[Sample text]
	\section{Glossary}% Oppure: Definitions, Acronyms, Abbreviations
		[Sample text]
	\section{Stakeholders}
		[Sample text]
	%\section{Proposed system}
	%	[Sample text]


	\chapter{Overall description}
    \section{Sample section}
		[Sample text]
    \chapter{Specific requirements}
    \section{Sample section}
		[Sample text]

	\part{Requirements specification}
	\chapter{Introduction}
    \section{Sample section}
		[Sample text]
	\chapter{Sample chapter}
	\section{Sample section}
		[Sample text]


    \newpage
    \appendix
    \chapter{Appendix A}
    \section{Sample element}

	\newpage
	\begin{thebibliography}{10}
		\bibitem{FirstReference}
			IEEE Std 830, Recommended Practice for Software Requirements Specifications, 1998
		\bibitem{SecondReference}
			Luca Mottola and Elisabetta Di Nitto, Software Engineering 2: Project goal, schedule and rules, 2016
	\end{thebibliography}

\end{document}
